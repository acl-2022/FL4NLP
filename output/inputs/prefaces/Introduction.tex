Due to increasing concerns and regulations about data privacy (e.g., General Data Protection Regulation), coupled with the growing computational power of edge devices, emerging data from realistic users have become much more fragmented, forming distributed private datasets across different clients (i.e., organizations or personal devices). Respecting users’ privacy and restricted by these regulations, we have to assume that users’ data in a client are not allowed to transfer to a centralized server or other clients. For example, a hospital does not want to share its private data (e.g., conversations, questions asked on its website/app) with other hospitals. This is despite the fact that models trained by a centralized dataset (i.e., combining data from all clients) usually enjoy better performance on downstream tasks (e.g., dialogue, question answering). Therefore, it is of vital importance to study NLP problems in such a scenario, where data are distributed across different isolated organizations or remote devices and cannot be shared for privacy concerns.

The field of federated learning (FL) aims to enable many individual clients to jointly train their models, while keeping their local data decentralized and completely private from other users or a centralized server. A common training schema of FL methods is that each client sends its model parameters to the server, which updates and sends back the global model to all clients in each round. Since the raw data of one client has never been exposed to others, FL is promising to be an effective way to address the above challenges, particularly in the NLP domain where many user-generated text data contain sensitive, personal information.


Our workshop website is \url{https://fl4nlp.github.io/}.